\documentclass[10pt,twocolumn,letterpaper]{article}

\usepackage{cvpr}
\usepackage{times}
\usepackage{epsfig}
\usepackage{graphicx}
\usepackage{amsmath}
\usepackage{amssymb}
\usepackage{subfig}
\usepackage[font={small}]{caption}
\usepackage{subcaption} 



% Include other packages here, before hyperref.

% If you comment hyperref and then uncomment it, you should delete
% egpaper.aux before re-running latex.  (Or just hit 'q' on the first latex
% run, let it finish, and you should be clear).
%\usepackage[pagebackref=true,breaklinks=true,letterpaper=true,colorlinks,bookmarks=false]{hyperref}

\cvprfinalcopy % *** Uncomment this line for the final submission

\def\cvprPaperID{****} % *** Enter the CVPR Paper ID here
\def\httilde{\mbox{\tt\raisebox{-.5ex}{\symbol{126}}}}

% Pages are numbered in submission mode, and unnumbered in camera-ready
\ifcvprfinal\pagestyle{empty}\fi
\begin{document}

%%%%%%%%% TITLE
\title{A SWT para detectar RBNs em corridas de rua}

\author{Gabriel Martins de Miranda\\
Universidade de Bras\'ilia\\
www.unb.br\\
{\tt\small gabrielmirandat@hotmail.com}
% For a paper whose authors are all at the same institution,
% omit the following lines up until the closing ``}''.
% Additional authors and addresses can be added with ``\and'',
% just like the second author.
% To save space, use either the email address or home page, not both
%\and
%Second Author\\
%Institution2\\
%irst line of institution2 address\\
%\small\url{http://www.author.org/~second}}
}

\maketitle
\thispagestyle{empty}

%\graphicspath{{'/home/gabriel/Downloads/1.DetectText-master/RELATORIO/1.THRESHOLD/'}}

%%%%%%%%% ABSTRACT
\begin{abstract}
   Automatizar o processo de reconhecimento de RBNs em corridas de rua ainda \'e um processo n\~ao resolvido. Grande parte 
   do tempo gasto por empresas de fotografias est\'a em etiquetar estas imagens, feito hoje manualmente. Apresentamos o poder 
   de uma SWT diferenciada capaz de detectar com precis\~ao larguras de tra\c{c}ados em imagens com caracteres de bordas finas
   (tags de corredores em corridas de rua). Do algoritmo temos um ponto de partida para poupar estas pessoas deste trabalho 
   manual cansativo e ma\c{c}ante.
\end{abstract}

%%%%%%%%% BODY TEXT
\section{Introdu\c{c}\~ao}

%B. Epshtein, E. Ofek, and Y. Wexler, “Detecting text in natural scenes with stroke width transform,” in Proc. Comput. Vision Pattern Recognit. ,2010, pp. 2963–2970.
%Ben-Ami, I., Basha, T. and Avidan, S., 2012. Racing Bib Numbers Recognition. In Proc of British Machine Vision Conference 2012. 19.1--19.10. 


Fot\'ografos (ou empresas ) que lidam com corridas de rua possuem a tarefa de etiquetar suas imagens para 
permitir acesso r\'apido a um determinado corredor. Automatizar este processo poderia poupar horas e horas de pobres pessoas
que trabalham de forma ma\c{c}ante num movimento estressante e repetitivo. Autores como Boris Epshtein \etal. proporam um 
brilhante ponto de partida em~\cite{Swtboris} para solucionar, de forma geral, problemas deste tipo, ou melhor, qualquer 
tipo de problema que envolva caracteres em imagens com cen\'arios naturais. Apesar deste grandioso ponto de partida, ainda 
hoje n\~ao \'e uma quest\~ao completamente resolvida. Em vez de propor uma solu\c{c}\~ao geral, n\'os pensamos de forma 
segmentada, ou seja, dirigir o que foi proposto em~\cite{Swtboris} para solucionar problemas deste tipo. Usaremos o termo proposto 
por Ben\-Ami \etal em seu trabalho descrito em~\cite{Rbniban}, Racing Bib Number Recognition (RBN) para nos dirigirmos \`as 
etiquetas utilizadas pelos corredores, cujo prop\'osito primeiro do artigo ser\'a detect\'a-las no maior n\'umero poss\'ivel destas 
fotografias. Observando imagens de corrida de rua, pudemos concluir algumas situa\c{c}\~oes que frequentemente ocorrem e da\'i 
formular nosso algoritmo com base nestes fatos: a RBN pode possuir algarismos falhados. As magnitudes das larguras dos 
algarismos numa RBN n\~ao s\~ao elevadas. H\'a grandes varia\c{c}\~oes de ilumina\c{c}\~ao presente numa mesma RBN. As RBNs 
podem estar dispostas em v\'arias orienta\c{c}\~oes. Podem haver RBNs de apenas 2 algarismos.

Diferente de~\cite{Rbniban}, que prop\~oe encontrar poss\'iveis regi\~oes em que h\'a maior probabilidade de se encontrar 
uma RBN para s\'o ent\~ao aplicar a SWT, nosso algoritmo \'e focado em direcionar a SWT para ganhar precis\~ao e velocidade 
de forma inovadora.

%------------------------------------------------------------------------
\section{Metodologia}

Para lidar com o problema de reconhecer RBNs e tentar alcan\c{c}ar o estado da arte, as seguintes decis\~oes foram tomadas:

1) Threshold para o detector de bordas

Os limites para o algoritmo de Canny, proposto em~\cite{Canny}, passaram de 175 para 300 no limite inferior, e de 320 para 
600 no limite superior. A mudan\c{c}a possui o objetivo principal de remover o ru\'ido caracter\'istico de RBNs falhadas.
Os valores SW nos algarismos com muitos detalhes/ru\'idos caracter\'isticos destas falhas acabam tornando-se muito variados 
o que ocasiona o descarte destes pelo processo. Como as RBNs possuem fundos bem destacados, como preto no branco, este mudan\c{c}a 
mant\'em apenas bordas relevantes.

 \vspace{2\baselineskip}\vspace{-\parskip}
 \begin{minipage}{\linewidth}
 \centering
 \scalebox{0.9}{\includegraphics[width=0.8\linewidth]{"../1.THRESHOLD/nat"}}
 \captionof{figure}{Observe esta RBN. Em tags de corridas de rua \'e normal este tipo de falhas.}
 \end{minipage}


\vspace{2\baselineskip}\vspace{-\parskip}
 \begin{minipage}{\linewidth}
 \centering
 \scalebox{1.15}{\includegraphics[width=0.8\linewidth]{"../1.THRESHOLD/img11"}}
 \captionof{figure}{(a) Ap\'os detecc\~ao de bordas, com thresholds entre 175 e 320. Perceba a quantidade de ru\'ido presente dentro dos algarismos
    (b) A transformada SW aplicada em (a). As falhas atrapalham consideravelmente o resultado da transformada.
    (c) Ap\'os descarte das BBs. Podemos ver que pedacos do algarismo foram descartados j\'a na etapa anterior.
    (d) Resultado final, ap\'os descarte de BBs e chaining. Veja que grande parte do algarismo foi descartado.
 }
 \end{minipage}
 
\vspace{2\baselineskip}\vspace{-\parskip}
 \begin{minipage}{\linewidth}
 \centering
 \scalebox{1.15}{\includegraphics[width=0.8\linewidth]{"../1.THRESHOLD/img22"}}
 \captionof{figure}{(a) Utilizando a mesma imagem, por\'em com os thresholds setados para 300 e 600. Observe que grande parte do ru\'ido que 
   estava presente dentro dos algarismos desapareceu.
    (b) A transformada SW aplicada em (a). Uma continuidade maior dos algarismos \'e obtida.
    (c) Mesmo ap\'os o descarte das BBs, vemos que o algarismo como um todo \'e mantido.
    (d) Resultado final, ap\'os descarte de BBs e chaining. Grande parte da RBN foi mantida.
 }
 \end{minipage}

%------------------------------------------------------------------------
2) Precis\~ao multiplicada pelas imagens gradiente.

A precis\~ao proposta, por ter um valor relativo elevado, adiciona elementos desnecess\'arios que posteriormente ser\~ao 
utilizados pela SWT como ponto de partida, podendo ocasionar ru\'idos pr\'oximos \`as RBNs. O que fizemos foi diminuir esta 
constante de 0.5 para 0.2.

  \vspace{2\baselineskip}\vspace{-\parskip}
 \begin{minipage}{\linewidth}
 \centering
 \scalebox{0.9}{\includegraphics[width=0.8\linewidth]{"../2.PRECISAO/nat"}}
 \captionof{figure}{Uma RBN comum.}
 \end{minipage}
 
 
 \vspace{2\baselineskip}\vspace{-\parskip}
 \begin{minipage}{\linewidth}
 \centering
 \scalebox{1.15}{\includegraphics[width=0.8\linewidth]{"../2.PRECISAO/img11"}}
 \captionof{figure}{(a) Ap\'os detecc\~ao de bordas com Canny.
    (b) A transformada SW aplicada em (a). Observe que um falso valor de tracado surge devido \`a constante de 0.5.
    (c) Ap\'os descarte das BBs. Vemos que o algarismo que apresenta aquela falso tracado foi descartado.
    (d) Resultado final, ap\'os descarte de BBs e chaining. 
 }
 \end{minipage}
 
\vspace{2\baselineskip}\vspace{-\parskip}
 \begin{minipage}{\linewidth}
 \centering
 \scalebox{1.15}{\includegraphics[width=0.8\linewidth]{"../2.PRECISAO/img22"}}
 \captionof{figure}{(a) Bordas atrav\'es de Canny.
    (b) A transformada SW aplicada em (a). Observe que aquele falso tracado desaparece quando utilizamos a constante de 0.2.
    (c) Mesmo ap\'os o descarte das BBs, vemos que todos os algarismo da RBN sao mantidos.
    (d) Resultado final, bastante satisfat\'orio.
 }
 \end{minipage}

%------------------------------------------------------------------------
3) Dist\^ancia das cores das CCs

Numa mesma RBN \'e comum que exista grandes varia\c{c}\~oes de luz, j\'a que grande parte das corridas ocorre \`a luz 
do dia e o local da tag reflete bastante pois frequentemente possui fundo branco. Pode acontecer tamb\'em que parte da 
tag fique sombreada, dificultando o reconhecimento se a restri\c{c}\~ao de cores para o chaining for muito elevado. Com 
isto, alteramos este valor de 1600 para 5000.
 
 \vspace{2\baselineskip}\vspace{-\parskip}
 \begin{minipage}{\linewidth}
 \centering
 \scalebox{0.9}{\includegraphics[width=0.8\linewidth]{"../5.COR/nat2"}}
 \captionof{figure}{Uma RBN comum. Perceba que o algarismo 5 se encontra mais sombreado que os outros.}
 \end{minipage}
 
 
 \vspace{2\baselineskip}\vspace{-\parskip}
 \begin{minipage}{\linewidth}
 \centering
 \scalebox{1.15}{\includegraphics[width=0.8\linewidth]{"../5.COR/img11"}}
 \captionof{figure}{(a) Ap\'os detecc\~ao de bordas com Canny.
    (b) A transformada SW aplicada em (a).
    (c) Ap\'os descarte das BBs.
    (d) Resultado final. Podemos ver aqui que o algarismo 5 foi descartado do n\'umero encontrado. Isto acontece pois este 
    algarismo est\'a mais sombreado que os outros da mesma RBN.
 }
 \end{minipage}
 
  
 \vspace{2\baselineskip}\vspace{-\parskip}
 \begin{minipage}{\linewidth}
 \centering
 \scalebox{1.15}{\includegraphics[width=0.8\linewidth]{"../5.COR/img22"}}
 \captionof{figure}{(a) Ap\'os detecc\~ao de bordas com Canny.
    (b) A transformada SW aplicada em (a).
    (c) Ap\'os descarte das BBs.
    (d) Resultado final, ap\'os descarte de BBs e chaining. Aqui, gracas a suavizac\~ao das dist\^ancias de cores, foi poss\'ivel 
    identificar o algarismo 5 como algarismo pertencente \`a RBN como um todo.
 }
 \end{minipage}
 
%-------------------------------------------------------------------------
4) \^Angulo entre CCs para chaining

Durante a corrida, muitas RBNs podem rotacionar no corpo do corredor. Com isto, o \^angulo entre os algarismos pode variar 
bastante. Pensando nisto, mudamos a restri\c{c}\~ao de que para se formar um chaining entre as BBs seria permitido no 
m\'aximo um \^angulo de 30 graus e passamos este \^angulo para 60 graus.

 \vspace{5.75\baselineskip}\vspace{-\parskip}
 \begin{minipage}{\linewidth}
 \centering
 \scalebox{0.9}{\includegraphics[width=0.8\linewidth]{"../6.ANGULO/nat2"}}
 \captionof{figure}{Uma RBN comum. Perceba que ela se encontra rotacionada.}
 \end{minipage}
 
 
 \vspace{5.75\baselineskip}\vspace{-\parskip}
 \begin{minipage}{\linewidth}
 \centering
 \scalebox{1.15}{\includegraphics[width=0.8\linewidth]{"../6.ANGULO/img11"}}
 \captionof{figure}{(a) Ap\'os detecc\~ao de bordas com Canny.
    (b) A transformada SW aplicada em (a).
    (c) Ap\'os descarte das BBs.
    (d) Resultado final. Como o algarismo 3 possui um certo \^angulo com os algarismo zeros, ele foi descartado.
 }
 \end{minipage}
 
  
 \vspace{2\baselineskip}\vspace{-\parskip}
 \begin{minipage}{\linewidth}
 \centering
 \scalebox{1.15}{\includegraphics[width=0.8\linewidth]{"../6.ANGULO/img22"}}
 \captionof{figure}{(a) Ap\'os detecc\~ao de bordas com Canny.
    (b) A transformada SW aplicada em (a).
    (c) Ap\'os descarte das BBs.
    (d) Resultado final, ap\'os descarte de BBs e chaining. Por permitir \^angulos de at\'e 60 graus, consegue-se unir o 
    algarismo 3 \'a sua RBN.
 }
 \end{minipage}
 
%-------------------------------------------------------------------------
5) N\'umero m\'inimo de BBs para que se forme uma chain

RBNs frequentemente possuem de 2 a 5 algarismos. Para tentar identificar a maioria das tags, mudamos a restri\c{c}\~ao 
para que seja necess\'ario um m\'inimo de duas BBs para que se forme uma chain. 

 \vspace{2\baselineskip}\vspace{-\parskip}
 \begin{minipage}{\linewidth}
 \centering
 \scalebox{0.9}{\includegraphics[width=0.8\linewidth]{"../7.COMPONENTS/nat2"}}
 \captionof{figure}{Uma RBN comum. Note que ela s\'o \'e formada por dois algarismos.}
 \end{minipage}
 
 
 \vspace{2\baselineskip}\vspace{-\parskip}
 \begin{minipage}{\linewidth}
 \centering
 \scalebox{1.15}{\includegraphics[width=0.8\linewidth]{"../7.COMPONENTS/img11"}}
 \captionof{figure}{(a) Ap\'os detecc\~ao de bordas com Canny.
    (b) A transformada SW aplicada em (a).
    (c) Ap\'os descarte das BBs.
    (d) Resultado final. Nada aparece, j\'a que n\~ao aconteceu chaining.
 }
 \end{minipage}
 
  
 \vspace{2\baselineskip}\vspace{-\parskip}
 \begin{minipage}{\linewidth}
 \centering
 \scalebox{1.15}{\includegraphics[width=0.8\linewidth]{"../7.COMPONENTS/img22"}}
 \captionof{figure}{(a) Ap\'os detecc\~ao de bordas com Canny.
    (b) A transformada SW aplicada em (a).
    (c) Ap\'os descarte das BBs.
    (d) Resultado final, ap\'os descarte de BBs e chaining. Por permitir um m\'inimo de duas BBs, o chaining ocorre e a 
    RBN n\~ao \'e descartada.
 }
 \end{minipage}
 
%------------------------------------------------------------------------- 
6) Valor mediano para o raio na segunda passada

Aqui escolhemos um valor menor que a mediana como SW para compor o raio. 
Isto homogeniza os valores dos tra\c{c}ados (caracter\'istica tamb\'em adquirida com a suaviza\c{c}\~ao na 
detec\c{c}\~ao de bordas) num mesmo algarismo e assim dificulta que ele seja rejeitado. Em vez de usarmos a mediana como valor SW 
de todo o raio, utilizamos o valor tamanho\_do\_vetor/12. Isto faz com que um valor SW menor em magnitude seja escolhido como 
o SW de todo o raio.
Esta mudan\c{c}a \'e interessante quando analizamos o algarismo 4. Por ter um cruzamento perpendicular, possui tanto 
larguras horizontais quanto verticais identificadas pela SWT original. O que acontece \'e que a segunda passada do algoritmo 
para homogeneizar estes valores encontra dificuldades no processo, e frequentemente este algarismo \'e rejeitado. Observe 
as imagens que se seguem:

 \vspace{2\baselineskip}\vspace{-\parskip}
 \begin{minipage}{\linewidth}
 \centering
 \scalebox{0.9}{\includegraphics[width=0.8\linewidth]{"../4.MEDIAN/numeros/nat"}}
 \captionof{figure}{Algarismos de 0 a 9.}
 \end{minipage}
 
 
 \vspace{2\baselineskip}\vspace{-\parskip}
 \begin{minipage}{\linewidth}
 \centering
 \scalebox{1.15}{\includegraphics[width=0.8\linewidth]{"../4.MEDIAN/numeros/img11"}}
 \captionof{figure}{(a) Ap\'os detecc\~ao de bordas com Canny.
    (b) A transformada SW aplicada em (a). Note que algumas descontinuidades ocorreram nos algarismos 0 e 1 devido 
    a falhas nas bordas. Por\'em, o algarismo 4 est\'a em perfeitas condicoes.
    (c) Ap\'os descarte das BBs. O algarismo 4, mesmo estando em boas condicoes, foi rejeitado. Isto acontece devido \`a 
    grandes diferencas dos valores SW nos raios que comp\~oem o 4.
    (d) Resultado final. O 2 e o 3 foram rejeitados pois inicialmente n\~ao havia formac\~ao de chain com apenas dois algarismos.
 }
 \end{minipage}
 
  
 \vspace{2\baselineskip}\vspace{-\parskip}
 \begin{minipage}{\linewidth}
 \centering
 \scalebox{1.15}{\includegraphics[width=0.8\linewidth]{"../4.MEDIAN/numeros/img22"}}
 \captionof{figure}{(a) Ap\'os detecc\~ao de bordas com Canny.
    (b) A transformada SW aplicada em (a). Perceba que agora as cores no algarismo 4 est\~ao distribu\'idas de forma mais 
    homog\^enea. 
    (c) Ap\'os descarte das BBs. Nenhum algarismo foi rejeitado no processo de formac\~ao das BBs. Note que devido ao 
    problema na detecc\~ao das bordas, os algarismos 0 e 1 ficaram unidos em uma s\'o BB.
    (d) Resultado final, ap\'os descarte de BBs e chaining. Todos os algarismos foram mantidos.
 }
 \end{minipage}

Vejamos na pr\'atica:

 \vspace{2\baselineskip}\vspace{-\parskip}
 \begin{minipage}{\linewidth}
 \centering
 \scalebox{0.9}{\includegraphics[width=0.8\linewidth]{"../4.MEDIAN/nat2"}}
 \captionof{figure}{Uma RBN com algarismos 4.}
 \end{minipage}
 
 
 \vspace{2\baselineskip}\vspace{-\parskip}
 \begin{minipage}{\linewidth}
 \centering
 \scalebox{1.15}{\includegraphics[width=0.8\linewidth]{"../4.MEDIAN/img11"}}
 \captionof{figure}{(a) Ap\'os detecc\~ao de bordas com Canny.
    (b) A transformada SW aplicada em (a).
    (c) Ap\'os descarte das BBs. Um dos algarismos 4 foi descartado.
    (d) Resultado final. Houve descarte de grande parte da tag.
 }
 \end{minipage}
 
  
 \vspace{2\baselineskip}\vspace{-\parskip}
 \begin{minipage}{\linewidth}
 \centering
 \scalebox{1.15}{\includegraphics[width=0.8\linewidth]{"../4.MEDIAN/img22"}}
 \captionof{figure}{(a) Ap\'os detecc\~ao de bordas com Canny.
    (b) A transformada SW aplicada em (a). Note que n\~ao se consegue perceber diferen\c{c}a com o caso em que se usa a 
    mediana.
    (c) Ap\'os descarte das BBs. Nenhum dos algarismos foi descartado.
    (d) Resultado final, ap\'os descarte de BBs e chaining.
 }
 \end{minipage} 

%-------------------------------------------------------------------------
7) Limitando o tamanho m\'aximo dos tra\c{c}ados da SWT

Os resultados apresentados na SWT original apresentam larguras de qualquer tamanho, dispostas de qualquer forma. Acontece que 
os algarismos presentes numa RBN tem um tamanho de largura do tra\c{c}ado extremamente limitado. Se considerarmos que uma RBN 
esteja disposta no peito de um corredor de forma que uma imagem tirada contenha o rosto dele na parte superior e a tag na inferior, 
de forma totalmente preenchida, podemos estimar que a largura dos algarismos n\~ao passam de 30 pixels. Diante disto, cortamos do resultado 
da SWT larguras acima de 30 pixels, e com isto obtivemos resultados interessant\'issimos. Al\'em de velocidade e remo\c{c}\~ao 
de ru\'ido, conseguimos detectar em v\'arios casos em que larguras n\~ao interessantes tornavam-se ru\'ido e impediam a defini\c{c}\~ao 
de BBs consistentes. Observe um exemplo:

 \vspace{2\baselineskip}\vspace{-\parskip}
 \begin{minipage}{\linewidth}
 \centering
 \includegraphics[width=0.8\linewidth]{"../3.POINTS_SIZE/natural2"}
 \captionof{figure}{Uma imagem de uma corrida.}
 \end{minipage}
 
  \vspace{2\baselineskip}\vspace{-\parskip}
 \begin{minipage}{\linewidth}
 \centering
 \includegraphics[width=0.8\linewidth]{"../3.POINTS_SIZE/stroke11"}
 \captionof{figure}{A SWT original aplicada na imagem. Observe a quantidade de informac\~ao que n\~ao nos \'e interessante. }
 \end{minipage}
 
  \vspace{2\baselineskip}\vspace{-\parskip}
 \begin{minipage}{\linewidth}
 \centering
 \includegraphics[width=0.8\linewidth]{"../3.POINTS_SIZE/stroke22"}
 \captionof{figure}{A SWT com restric\~ao nos tracados. O que resta s\~ao apenas raios relevantes para a detecc\~ao de RBNs.}
 \end{minipage}
 
 Olhando mais de perto, temos:
 
 \vspace{2\baselineskip}\vspace{-\parskip}
 \begin{minipage}{\linewidth}
 \centering
 \scalebox{1.15}{\includegraphics[width=0.8\linewidth]{"../3.POINTS_SIZE/img11"}}
 \captionof{figure}{(a) Ap\'os detecc\~ao de bordas com Canny.
    (b) A transformada SW aplicada em (a). Observe a proximidade de raios irrelevantes com os algarismos da RBN.
    (c) Ap\'os descarte das BBs. O algarismo 1 foi rejeitado pois n\~ao foi poss\'ivel criar uma BB que o envolvesse.
    (d) Resultado final, com a tag incompleta.
 }
 \end{minipage}
 
  
 \vspace{2\baselineskip}\vspace{-\parskip}
 \begin{minipage}{\linewidth}
 \centering
 \scalebox{1.15}{\includegraphics[width=0.8\linewidth]{"../3.POINTS_SIZE/img22"}}
 \captionof{figure}{(a) Ap\'os detecc\~ao de bordas com Canny.
    (b) A transformada SW aplicada em (a). O raio irrelevante sumiu devido \`a restric\~ao da largura m\'axima.
    (c) Ap\'os descarte das BBs. Vemos que aqui nenhum dos algarismos da tag \'e rejeitado.
    (d) Resultado final, ap\'os descarte de BBs e chaining. Nada \'e perdido e obtemos um resultado satisfat\'orio.
 }
 \end{minipage}
 
\section{Resultados}

Num banco de 132 imagens de uma mesma corrida, obtivemos reconhecimento de todas as RBNs em 87.93\% das imagens, contra 65.52\% 
utilizando a SWT original. Das 14 imagens n\~ao reconhecidas, tinhamos imagens borradas e RBNs de tamanho muito pequeno. 16 das 
imagens eram irreconhec\'iveis devido \`a oclus\~ao de parte da RBN ou n\~ao presen\c{c}a de nenhuma RBN.

Num banco de 125 imagens de corridas diferentes, obtivemos o resultado de 75.60\% , contra 47.15\% utilizando a SWT original.
Das 30 n\~ao reconhecidas, uma estava relativamente borrada, 9 possu\'iam grande mudan\c{c}a de ilumina\c{c}\~ao na RBN e as outras 
20 possuiam RBN com pouco contraste com o fundo. 

O desempenho tamb\'em melhorou bastante. Contabilizando o tempo total para rodar ambos os algoritmos no banco de imagens completo 
com 257 imagens, nossa abordagem levou 4 minutos e 40 segundos contra 20 minutos e 48 segundos com a SWT original. 

\begin{table}[bp]
\centering
\resizebox{\textwidth}{!}{%
\begin{tabular}{|l|l|l|l|l|l|l|l|l|l|}
\hline
                & \multicolumn{4}{l|}{Banco 1 - 132 imagens}                    & \multicolumn{4}{l|}{Banco 2 - 125 imagens}                    & Banco total - 257 imagens \\ \hline
                & Identificados & N\~ao identificados & Irreconhec\'iveis & Acerto  & Identificados & N\~ao identificados & Irreconhec\'iveis & Acerto  & Tempo gasto               \\ \hline
Nosso algoritmo & 102           & 14                & 16              & 87.93\% & 93            & 30                & 2               & 75.60\% & 4 min 40 seg              \\ \hline
SWT original    & 76            & 40                & 16              & 65.52\% & 58            & 65                & 2               & 47.15\% & 20 min 48 seg             \\ \hline
\end{tabular}
}
\caption{Comparando nosso algoritmo com a SWT original.}
\label{my-label}
\end{table}

\section{Conclus\~ao}

Neste trabalho mostramos como utilizar a Stroke Width Transform numa abordagem em que n\~ao se possui interesse em caracteres 
de tamanho de largura elevado, no caso, em RBNs de corridas de rua. Atrav\'es da restri\c{c}\~ao do tamanho m\'aximo de 
largura, conseguimos melhorar os resultados da SWT original, assim como aumentar a velocidade do algoritmo.
Das imagens n\~ao reconhecidas, tinhamos imagens borradas, tags de tamanho muito pequeno e baixo contraste dos algarismos 
com o fundo. \'E importante salientar que este \'ultimo torna-se um problema maior devido \'a suaviza\c{c}\~ao dos filtros 
de borda e da constante de precis\~ao multiplicada pelas imagens gradiente. Al\'em disto, apesar de termos minimizado a 
quantidade de ru\'ido, ele surgiu de outras formas dadas as modifica\c{c}\~oes, constituindo um vi\'es do algoritmo. 
Sugerimos que trabalhos futuros se empenhem em corrigir estes problemas.
 
 
{\small
\bibliographystyle{ieee}
\bibliography{egbib}
}

\end{document}
